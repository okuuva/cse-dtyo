%
% Template for Department of Electrical and Information Engineering Diploma Thesis v1.1.2013
% Authors: Mika Korhonen (original author), Pekka Pietikäinen, Christian Wieser, Teemu Tokola and Juha Kylmänen.
% If you make any improvements to this template, please contact ouspg@ee.oulu.fi
%

\documentclass[a4paper, 12pt,titlepage]{dithesis}
\usepackage[english,finnish]{babel}
\usepackage[utf8]{inputenc}
\usepackage[T1]{fontenc}  
\usepackage{times}
\usepackage{tabularx}
\usepackage{graphicx}
\usepackage{float}
\usepackage{enumerate}
\usepackage{placeins}
\usepackage{fancybox}
\usepackage{verbatim}
\usepackage{longtable}
\usepackage{di}
\usepackage[hyphens]{url}
\usepackage{boxedminipage}
\usepackage{subfigure}
\usepackage{multirow}
\usepackage{amsfonts}
\usepackage{xcolor}
\tolerance=500

%\usepackage[a4paper,margin=2.5cm,dvips]{geometry}
%\geometry{papersize={210mm,297mm}}
%dvipdf -sPAPERSIZE=a4

% The following code removes %-signs with URL:s longer than 72 chars
\begingroup
\makeatletter
\g@addto@macro{\UrlSpecials}{%
  \endlinechar=13 \catcode\endlinechar=12
  \do\%{\Url@percent}\do\^^M{\break}}
 \catcode13=12 %
 \gdef\Url@percent{\@ifnextchar^^M{\@gobble}{\mathbin{\mathchar`\%}}}%
\endgroup %

%\selectlanguage{finnish}

\otsikko{Linux-ydin laajennosten tunnistus staattisen binäärianalyysin avulla}
\title{Linux kernel module recognition using static binary analysis}

\etunimi{Oula}
\sukunimi{Kuuva}
\valvoja{prof. Juha Röning}
\koulutusohjelma{information} % {information | electrical}
\vuosi{2017}
\tyo{Master} % {Bachelor | Master}
\kieli{english} % {finnish | english}

\begin{document}

\begin{titlepage}
	\centering{\includegraphics*[width=0.3\textwidth]{uni_logo}\\}
	{\sffamily\fontsize{9}{1pt}\selectfont FACULTY OF INFORMATION TECHNOLOGY AND ELECTRICAL ENGINEERING\\}
	%{\sffamily\fontsize{9}{1pt}\selectfont TIETO- JA SÄHKÖTEKNIIKAN TIEDEKUNTA\\}
	\vspace{65 mm}
	{\textbf{\fontsize{16}{19pt}\selectfont \getfirstname\ \getlastname }\\}
	\vspace{15 mm}
	{\textbf{\fontsize{18}{22pt}\selectfont LINUX KERNEL MODULE RECOGNITION\\USING STATIC BINARY ANALYSIS\\}}
	\vspace{60 mm}
	{\fontsize{14}{17}\selectfont Master's Thesis \\Degree Programme in Computer Science and Engineering \\ Month 2017\\}
\end{titlepage}

\selectlanguage{english}

\begin{abstract}
This is a sample abstract.

\keywords Linux, loadable kernel module, software composition analysis, static binary analysis

\end{abstract}

\selectlanguage{finnish}
\begin{tiivistelma}
Esimerkkitiivistelmä

\avainsanat Linux, järjestelmäydinlaajennos, ohjelmistokoostumusanalyysi, staattinen binäärianalyysi
\end{tiivistelma}

\selectlanguage{english}

\sisluettelo
%\tableofcontents

\otsake{FOREWORD}
This \LaTeX -template has been used by various people at department
since the late 1990's, and has slowly improved over time.  It is still
somewhat rough at the edges, but hopefully will be helpful in reducing
some of the pain involved in writing a diploma thesis.

Contributors to the template include Mika Korhonen (original author),
Pekka Pietikäinen, Christian Wieser and Teemu Tokola.  If you make any
improvements to this template, please contact ouspg@ee.oulu.fi, and we
will try to include them in further revisions.

The template was updated during the summer of 2013 by Juha Kylmänen.
%\allekirjoitus{Oulu, Finland \today}

\otsake{ABBREVIATIONS}

\setlongtables
\begin{longtable}[l]{p{3cm}p{0.7\textwidth}}

% Add your abbreviations to abbreviations.tex
LKM & loadable kernel module\\
SUT & system under testing\\
NVD & national vulnerability database\\
ELF & executable and linkable format\\
SCA & software composition analysis\\
SBA & static binary analysis\\
OSS & open source software\\
IL & intermediate languare\\
OS & operating system\\


\end{longtable}
\setcounter{table}{0}

\chapter{Introduction}
\sivunumerot
% Introduction Chapter

\section{Background and motivation}

Modern software development isn't about writing everything from scratch, it's more about selecting
the right ready-made pieces and gluing them together. As the use of open-source software (OSS) is
on a steady rise even in the proprietary software products \cite{deshpande2008total}, the attack
surface of the programs has too increased from the in-house code to the third-party components.

Linux is the most widely used operating system (OS) in the world \cite{osmarketshare}\footnotemark.
It is used everywhere from tiniest embedded systems \cite{picotux} to the biggest super computers
\cite{top500linuxshare}. It also contains a notable number of known vulnerabilities
\cite{cvedetailslinuxkernel}, holding at the time of writing the somewhat questionable title of the
most vulnerable product according to CVE Details \cite{cvedetailstop50}. However, many of the
reported vulnerabilities actually affect some drivers, network stack or filesystem handlers instead
of the kernel core itself. These subcomponents of the kernel are usually packed as kernel modules
(LKM) \cite{whatislkm}, meaning that all vulnerabilities incorrectly reported for the kernel are
invalid if the system doesn't contain the actual vulnerable modules.

Should I have a third paragraph here? And if so what would it be about?

\footnotetext{Android OS is based on Linux. Does this need a source? And can a footnote have a cite?}

\section{Research objectives}

The main emphasis of this thesis is to identify what LKM are included with the Linux kernel in a
given input binary. This is achieved by generating a unique signature for each LKM. These
signatures are then seached from the given input binary. If large enough proportion of a signature
is found from the input binary, then that LKM is concidered to be found from it. A Python
implementation is developed for this purpose.

The system has two phases, teaching phase where unique signatures are being generated from the LKM
and saved to the database and scanning phase where signatures extracted from the input binary are
being compared against those in the database to determine what known LKM, if any, are part of the
said input binary.

The signature extraction and matching utilises the same algorithm developed for Protecode Supply
Chain, previously known as Codenomicon AppCheck, developed by Synopsys Finland Ltd. This
implementation was chosen as it has previously proven to be very effective and efficient in finding
other open-source software (OSS) components from binary input \cite{vayrynen2014finding}.

The secondary objective is to be able to detect what included LKM are loaded in runtime using
static analysis only. The input binaries might contain some excess LKM that aren't actually loaded
at runtime. Being able to detect what included LKM are really in use in the given input binary at
least to some extent could bring the number of false positives in vulnerability reporting down even
further and give users a hint about possibly misconfigured kernel in case of large number of unused
LKM in the input binary.

One big concern is the quality of vulnerability data of Linux kernel and LKM in National
Vulnerability Database (NVD). Given that so many of the vulnerabilities actually affecting the LKM
are being incorrectly reported to affect the Linux kernel core, it might be impractical to use the
NVD data feed as is to pinpoint the vulnerabilities to correct LKM without curation. As the amount
of NVD entries tied to Linux kernel, such a massive data curation operation is out of the scope of
this thesis.

Given the emphasis of the thesis, the research questions are the following:

\label{sect:questions}
\begin{itemize}

\item Question 1

\item Question 2

\end{itemize}

 % ./introduction.tex

\chapter{State of the Art}
% State of the Art Chapter

\section{Software Composition Analysis}

As the third party software and OSS specifically has become a critical part of practically every
bigger software project, commercial and proprietary products included \cite{blah}, it has become
increasingly difficult to keep track of what third party components are included in a given
project. This is quite problematic, as the OSS components included can contain security flaws that
can cause serious harm for the companies and their customers in the form of data loss, leaking
intellectual property and so forth. Luckily there are automatic tools available that can keep track
of third party components included in the project and notify the developers when new
vulnerabitilies are reported in the used components and when there's updates available
\cite{pittenger2016know}.

Software composition analysis has steadily raised its profile over the years \cite{blah}. While it
still hasn't quite reached a mainstream status yet, it has been made available for the masses by
actors such as Docker with the introduction of Docker Security Scan in Docker Hub and Docker Cloud
hosting services \cite{dockerscan}.

There are multiple commercial SCA solutions available from providers such as Black Duck Software,
\cite{blackduckhub}, Veracode \cite{veracodesca}, WhiteHat security \cite{whitehatsentinel},
Whitesource \cite{whitesourceosi} and Synopsys \cite{synopsysprotecode}. The main purpose of these
products is to report license information and known vulnerabilities of the identified third party
libraries from the system under test. There are multiple different ways to identify the third party
components. One common way is to statically analyse build files from the source code repository and
deduce what third party libraries are included when the project is compiled. Another way is to try
to figure out what third party library the analyzed code belongs to.

\subsection{Static Binary Analysis}

Static binary analysis is conducted by analyzing the compiled binary without actually executing it.
Working with the binary has some upsides compared to source code: by analyzing the binary we are
analyzing the very same thing the end users are using, thus eliminating the possibility of some
false positives which could result in source code analysis when some or all third party component
dependencies are included using dynamic links by the compiler. There is also the obvious benefit of
not needing to have an access to the source code, which is usually hard or impossible with
proprietary software. However, like Song et al. point out \cite{song2008bitblaze}, there are some
serious down sides to it as well, like additional complexity compared to source code analysis and
making the analysis architecture independent, i.e. supporting binaries compiled for different
processor architectures such as ARM, x86 and MIPS.

\section{Linux Kernel}

Blah.

\subsection{Linux Kernel Modules}

Blah.
 % ./sota.tex

\chapter{Design and Implementation}
% Implementation Chapter

Your implementation.
  % ./design-and-implementation.tex

\chapter{Evaluation}
% Evaluation Chapter

This chapter answers the research questions introduced in the chapter \ref{sect:questions}:

\begin{itemize}

\item Question 1

\item Question 2

\end{itemize}


\section{Evaluation outline}

The evaluation utilizes a set of compiled Linux LKMs and several Linux based firmware images. The
LKM binaries are used in the learning phase to teach the unique signatures of each module to the
scanner. The firmware images with known kernel configuration are then scanned to test the detection
rate of the new signatures.

Three sets of results are gathered. First only the modules in the firmwares are taught to the
scanner and the firmwares are scanned. Looking at the results we can determine whether the small
footprint of the modules is enough to generate viable signatures that can be reliably detected from
the firmware binaries.

In the second phase the same firmwares are compiled for different processor architecture and
scanned in order to see if the signatures are platform independent.

In the third phase involves mass teaching of different modules and rescanning the firmwares to see
if the added number of signatures reduce the size of unique signatures so much that the scan
results become unreliable due to growing number of false positives and false negatives.

The performance of the system is being measured by comparing the list of identified Linux LKMs from
the firmware binary to the list of LKMs known to be present in the binary.

\section{Test setup}

Blah.
  % ./evaluation.tex

\chapter{Discussion}
\input{discussion}  % ./discussion.tex

\chapter{Conclusion}
\input{conclusion}  % ./conclusion.tex

\bibliographystyle{di}
\bibliography{di}
\end{document}
