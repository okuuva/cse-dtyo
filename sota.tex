% State of the Art Chapter

\section{Software Composition Analysis}

Software composition analysis has steadily raised its profile over the years \cite{blah}. While it
still hasn't quite reached a mainstream status yet, it has been made available for the masses by
actors such as Docker with the introduction of Docker Security Scan in Docker Hub and Docker Cloud
hosting services \cite{dockerscan}.

Here be some academics.

There are multiple commercial SCA solutions available from providers such as Black Duck Software,
\cite{blackduckhub}, Veracode \cite{veracodesca}, WhiteHat security \cite{whitehatsentinel},
Whitesource \cite{whitesourceosi} and Synopsys \cite{synopsysprotecode}. The main purpose of these
products is to report license information and known vulnerabilities of the identified third party
libraries from the system under test. There are multiple different ways to identify the third party
components. One common way is to statically analyse build files from the source code repository and
deduce what third party libraries are included when the project is compiled. Another way is to try
to figure out what third party library the analyzed code belongs to.

\subsection{Static Binary Analysis}

Static binary analysis is conducted by analyzing the compiled binary without actually executing it.

\section{Linux Kernel}

Blah.

\subsection{Linux Kernel Modules}

Blah.

