% Introduction Chapter

\section{Background and motivation}

Modern software development isn't about writing everything from scratch, it's more about selecting
the right ready-made pieces and gluing them together. As the use of open-source software (OSS) is
on a steady rise even in the proprietary software products \cite{deshpande2008total}, the attack
surface of the programs has too increased from the in-house code to the third-party components.

Linux is by far the most widely used operating system (OS) in the world \cite{blah}. It is used
everywhere from tiniest embedded systems \cite{picotux} to the biggest super computers
\cite{top500linuxshare}. It also contains a notable number of known vulnerabilities
\cite{cvedetailslinuxkernel}, holding at the time of writing the somewhat questionable title of the
most vulnerable product according to CVE Details \cite{cvedetailstop50}. However, many of the
reported vulnerabilities actually affect some drivers, network stack or filesystem handlers instead
of the kernel core itself. These subcomponents of the kernel are usually packed as kernel modules
(LKM) \cite{blah}, meaning that all vulnerabilities incorrectly reported for the kernel are invalid
if the system doesn't contain the actual vulnerable modules.

Should I have a third paragraph here? And if so what would it be about?

\section{Research objectives}

The main emphasis of this thesis is to identify what LKM are included with the Linux kernel in a
given input binary. This is achieved by generating a unique signature for each LKM. These
signatures are then seached from the given input binary. If large enough proportion of a signature
is found from the input binary, then that LKM is concidered to be found from it. A Python
implementation is developed for this purpose.

Given the emphasis of the thesis, the research questions are the following:

\label{sect:questions}
\begin{itemize}

\item Can Linux loadable kernel modules be identified from the firmware image binary using same or
    similar signature extracting method that is beign used for kernel recognition?

\item Can static binary analysis be used in detecting what included kernel modules are loaded in runtime?

\item Can improved granularity in Linux kernel recognition be used in reducing false positives in
    vulnerability reporting?

\end{itemize}

