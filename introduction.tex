% Introduction Chapter

\section{Background and motivation}

Modern software development isn't about writing everything from scratch, it's more about selecting
the right ready-made pieces and gluing them together. As the use of open-source software (OSS) is
on a steady rise even in the proprietary software products \cite{deshpande2008total}, the attack
surface of the programs has too increased from the in-house code to the third-party components.

Linux is the most widely used operating system (OS) in the world \cite{osmarketshare}\footnotemark.
It is used everywhere from tiniest embedded systems \cite{picotux} to the biggest super computers
\cite{top500linuxshare}. It also contains a notable number of known vulnerabilities
\cite{cvedetailslinuxkernel}, holding at the time of writing the somewhat questionable title of the
most vulnerable product according to CVE Details \cite{cvedetailstop50}. However, many of the
reported vulnerabilities actually affect some drivers, network stack or filesystem handlers instead
of the kernel core itself. These subcomponents of the kernel are usually packed as kernel modules
(LKM) \cite{whatislkm}, meaning that all vulnerabilities incorrectly reported for the kernel are
invalid if the system doesn't contain the actual vulnerable modules.

Should I have a third paragraph here? And if so what would it be about?

\footnotetext{Android OS is based on Linux. Does this need a source? And can a footnote have a cite?}

\section{Research objectives}

The main emphasis of this thesis is to identify what LKM are included with the Linux kernel in a
given input binary. This is achieved by generating a unique signature for each LKM. These
signatures are then seached from the given input binary. If large enough proportion of a signature
is found from the input binary, then that LKM is concidered to be found from it. A Python
implementation is developed for this purpose.

The system has two phases, teaching phase where unique signatures are being generated from the LKM
and saved to the database and scanning phase where signatures extracted from the input binary are
being compared against those in the database to determine what known LKM, if any, are part of the
said input binary.

The signature extraction and matching utilises the same algorithm developed for Protecode Supply
Chain, previously known as Codenomicon AppCheck, developed by Synopsys Finland Ltd. This
implementation was chosen as it has previously proven to be very effective and efficient in finding
other open-source software (OSS) components from binary input \cite{vayrynen2014finding}.

The secondary objective is to be able to detect what included LKM are loaded in runtime using
static analysis only. The input binaries might contain some excess LKM that aren't actually loaded
at runtime. Being able to detect what included LKM are really in use in the given input binary at
least to some extent could bring the number of false positives in vulnerability reporting down even
further and give users a hint about possibly misconfigured kernel in case of large number of unused
LKM in the input binary.

One big concern is the quality of vulnerability data of Linux kernel and LKM in National
Vulnerability Database (NVD). Given that so many of the vulnerabilities actually affecting the LKM
are being incorrectly reported to affect the Linux kernel core, it might be impractical to use the
NVD data feed as is to pinpoint the vulnerabilities to correct LKM without curation. As the amount
of NVD entries tied to Linux kernel, such a massive data curation operation is out of the scope of
this thesis.

Given the emphasis of the thesis, the research questions are the following:

\label{sect:questions}
\begin{itemize}

\item Question 1

\item Question 2

\end{itemize}

