% Evaluation Chapter

This chapter answers the research questions introduced in the chapter \ref{sect:questions}:

\begin{itemize}

\item Question 1

\item Question 2

\end{itemize}


\section{Evaluation outline}

The evaluation utilizes a set of compiled Linux LKMs and several Linux based firmware images. The
LKM binaries are used in the learning phase to teach the unique signatures of each module to the
scanner. The firmware images with known kernel configuration are then scanned to test the detection
rate of the new signatures. NVD is utilized in mapping known vulnerabilities of the kernel modules
to recognized components.

Two sets of results are gathered. First only the modules known to be included in the test
firmwares are taught to the scanner and the firmwares are scanned. Looking at the results we can
determine whether the small footprint of the modules is enough to generate viable signatures that
can be reliably detected from the firmware binaries.

The second phase involves mass teaching of different modules and rescanning the firmwares to see
if the added number of signatures reduce the size of unique signatures so much that the scan
results become unreliable due to growing number of false positives and false negatives.

The performance of the system is being measured by comparing the list of identified Linux LKMs from
the firmware binary to the list of LKMs known to be present in the binary. The number of reported
vulnerabilities in the scan results is also compared before and after mapping the vulnerabilities to
LKMs affected.

\section{Test setup}

Blah.
